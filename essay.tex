\documentclass{article}

\title{Challenges when scaling the Agile Methodology and how to Overcome them}
\date{2017-11-11}
\author{1604281}

\begin{document}

\maketitle
\pagenumbering{arabic}

\paragraph{Abstract}


\section{Introduction}

Agile development methodologies are gaining popularity in the software industry, (FIND REF) and are already used by many companies. However, Agile Software Development is not without its drawbacks. One of the most common criticisms of the agile methodology \cite{begel2007usage} is the difficulties encountered when up-scaling the size of Agile projects. Although this paper is addressing general Agile practices and problems, the Agile methodology is also used in the games industry (FIND REF) which can range from small independent studios to massive teams working on multiple pieces of software. As such, the potential to scale Agile development to suit the nature of the game development environment could be beneficial to the games industry. 

\section{Challenges}

Though there are many issues involved in scaling an Agile project \cite{turk2014limitations}, the two that will be addressed in this paper are Team Size and Project Complexity. 

\paragraph{Project Complexity}\mbox{}\newline

As a project becomes larger and more complex, several issues arise when approaching it with the Agile methodology. Firstly, the Agile principle of ``Welcome changing requirements, even late in development." (REF), while feasible to implement in a small project, becomes increasingly difficult as the software becomes more complex. For example, if a project has many constituent parts which each have many dependencies on each other, changing one aspect of the project late in development may cause problems in other parts of the project. This can be compounded by another aspect of project complexity - mixing of Agile and Non-Agile teams. \\

A large project will inevitably have multiple teams working on different aspects of it concurrently, often leading to both Agile and Non-Agile groups collaborating. This can cause clashes in design approach and lead to some groups not fully understanding the design of the project. 

 

\paragraph{Team Size}\mbox{}\newline

Team size issues go here

\section{Solutions}


\section{Conclusion}



%%\nocite{*}
\bibliography{references}
\bibliographystyle{ieeetr}


\end{document}